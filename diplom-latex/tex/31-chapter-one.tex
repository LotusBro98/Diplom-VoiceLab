\chapter{Обзор существующих подходов и теоретическая справка}
\label{cha:ch_1}


\section{Архитектуры нейронных сетей для синтеза речи}
 - какие есть
   - диффузия
   - трансформер
   - unet для замены голоса
   - wav2vec
   - одностадийные, двухстадийные
 - понятие спектрограммы
 - примеры, ссылки
 - проблемы для каждой
 - сравнение, преимущества и недостатки, сферы применения

\section{Cуществующие вокодеры}
 - Существующие Вокодеры
  - GriffinLim (algo) \cite{1164317}
  - WaveGlow (gan)
  - EnCodec
 - Приемущества и недостатки существующих вокодеров 
 - примеры генерации от вокодеров, проблемы с ними
 - Алгоритм Гриффина-Лима
  - верхнеуровневое описание
 - нейросеть привязана к домену, на котором училась, алгоритм не привязан, но могут быть артефакты
 - быстродействие алгоритма и нейросети

рисунок с котиком \ref{fig:test}

\section{Представление и анализ аудиосигналов}
 - оконное преобразование Фурье
 - принцип неопределенности в пространстве частота-время
 - окна
 - децибел шкала амплитуд
   - почему такая
   - формула
   - порог слышимости
 - мел шкала
   - почему такая
   - формула
 - выбор оптимальной шкалы частот
  - линейная шкала
  - логарифмическая шкала
  - мел шкала
  - комбинированная шкала
 - избыточность информации в спектрограмме
 - наложение по времени
 - наложение по частоте
 - оптимальный размер окна для конкретной шкалы частот
 - сравнение с вейвлетами
 - восстановление сигнала с наложением (суммирование, сумма дает 0 там где надо, но хвосты могут расползаться)
 - оптимальная форма окна для устойчивости к искажениям
   - минимизация связанности далеких данных
 - Алгоритм Гриффина-Лима
    - как работает

 \section{Устойчивость к трансформациям и артефактам}
 - трансформации спектрограммы, устойчивость вокодера к трансформациям
   - растяжение\сжатие по времени
   - перемещение по шкале частот
   - вырезание\вставка куска
   - суммирование сигналов
 - от чего появляются трансформации и зачем нужны
 - артефакты от нейросетей
   - примеры полосок от сверток
   - BatchNorm паразитный bias
   - потеря шума
   - потеря outliers
 - устойчивость вокодеров к артефактам
 - инвариантность к сдвигу по времени, когда есть, зачем нужна

\section{Открытые проблемы}
 - борьба за качество генерации
 - как восстанавливать сигнал с нелинейной шкалой частот, когда нет ортогонального базиса
 - нет алгоритма, который был бы достаточно устойчив к артефактам от нейросетей
 - нейросетевые вокодеры:
   - требуют вычислительных ресурсов
   - нужно обучать на больших объемах данных
   - часто завязаны на домен, для другого домена могут хуже работать
 - избыточность представлений, тащим лишнюю информацию
 - нужна устойчивость обучения, глубокие сети сложнее учить

\begin{figure}[t]
  \centering
  \includegraphics[width=16cm]{figures/cat}
  \caption{test test test}
  \label{fig:test}
\end{figure}


\section{Выводы по главе}

