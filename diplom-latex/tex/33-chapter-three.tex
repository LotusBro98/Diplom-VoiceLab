\chapter{Полученные результаты}

\section{Результаты экспериментов}
\label{sec:sec_3_1}

\subsubsection{Автоэнкодер представления с сохранененной фазой}
\label{subsubsec:keep_phase_net}

\begin{markdown}
  - картинки со спектрограммами, сравнение
  - эксперименты
   - прямой и обратный проход, 
     - разница спектрограмм, 
     - разница сигналов, 
     - человеческое восприятие
     - с реконструкцией фазы и с сохранением
     - трансформации и устойчивость к артефактам
   - самодельный автоэнкодер 
     - с сохранением фазы
     
	 - с реконструкцией фазы
   - text to speech сети
     - FastSpeech2
\end{markdown}

\section{Наблюдения}
\begin{markdown}
  - рекомендуемые значения параметров
  - как устроена картинка с фазами, когерентный сигнал крутится вдоль времени с частотой сигнала, его след крутится весь на той же самой частоте
  - избыточность информации в шумах, можно заменить сам шум на информацию о распределении шума
  - голос и волны от резонаторов можно разложить на f0 (pitch) и АЧХ резонатора, и перемножать гармоники на АЧХ резонатора, можно будет легко менять pitch.
\end{markdown}

\section{Практическая ценность алгоритма}
\begin{markdown}
   сферы применения алгоритма, польза от него:
    - польза для нейросетей
      - скорость инференса
      - скорость обучения
      - стабильность обучения
      - простота разработки
      - без потери качества
  - сравнение метода с другими вокодерами
    - удобство
    - гибкость
    - универсальность
    - точность
    - скорость
  - примеры работы этого алгоритма и других, сравнение, подсветить недостатки других
\end{markdown}

\section{Планы развития}
\begin{markdown}
 - дальнейшая работа
   - разделение на когерентный сигнал и шум (в шуме хранится больше информации чем нужно, нужно сохранять только распределение, а не реализацию)
   - голос и волны от резонаторов можно разложить на f0 (pitch) и АЧХ резонатора, и перемножать гармоники на АЧХ резонатора, можно будет легко менять pitch.
   - реализация быстрого алгоритма на ускорителях
   - внедрение в существующие модели text-to-speech
\end{markdown}

\section{Выводы по главе}