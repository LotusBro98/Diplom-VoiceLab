\chapter{Полученные результаты}














\section{План эксперимента для оценки качества декодеров для синтеза речи}

\section{Цель эксперимента}

Цель эксперимента — сравнение качества синтеза речи с использованием различных конфигураций FastSpeech 2 и декодеров спектрограмм. В частности, будет проверена эффективность декодеров в сочетании с моделями синтеза речи, использующими различные форматы спектрограмм.

\section{Эксперимент 1: FastSpeech 2 (оригинал) + HiFi-GAN}

В рамках первого эксперимента используется стандартная версия модели FastSpeech 2, которая обучена на обычной mel-спектрограмме. Для декодирования спектрограммы применяется HiFi-GAN. Ожидается, что эта конфигурация покажет стабильные результаты, так как HiFi-GAN является одним из лучших декодеров для спектрограмм.

\subsection{Шаги:}
\begin{enumerate}
    \item Использование стандартной mel-спектрограммы для обучения модели FastSpeech 2.
    \item Обучение FastSpeech 2 с использованием оригинальной архитектуры.
    \item Использование HiFi-GAN для декодирования выходных спектрограмм FastSpeech 2.
    \item Оценка качества синтезированного аудио с использованием метрик PESQ, STOI, MOS.
\end{enumerate}

\section{Эксперимент 2: FastSpeech 2 (финетюнен на мой формат) + мой декодер}

Во втором эксперименте модель FastSpeech 2 будет адаптирована для работы с изменённым форматом спектрограммы (мел в нижней области частот и линейный спектр начиная с 4000 Гц). Для декодирования будет использован разработанный собственный декодер.

\subsection{Шаги:}
\begin{enumerate}
    \item Подготовка обучающих данных с использованием нового формата спектрограммы.
    \item Модификация FastSpeech 2 для работы с новым форматом спектрограммы (150 каналов).
    \item Обучение FastSpeech 2 на новом формате спектрограммы.
    \item Применение собственного декодера для восстановления аудио.
    \item Оценка качества синтезированного аудио с использованием метрик PESQ, STOI, MOS.
\end{enumerate}

\section{Эксперимент 3: FastSpeech 2 (финетюнен на Griffin-Lim) + Griffin-Lim}

В третьем эксперименте будет использована традиционная методика восстановления аудио с помощью Griffin-Lim, применённая к mel-спектрограмме. Это обеспечит контрольный показатель для сравнения с новыми подходами.

\subsection{Шаги:}
\begin{enumerate}
    \item Использование стандартной mel-спектрограммы для обучения модели FastSpeech 2.
    \item Обучение FastSpeech 2 с использованием Griffin-Lim.
    \item Применение Griffin-Lim для восстановления аудио.
    \item Оценка качества синтезированного аудио с использованием метрик PESQ, STOI, MOS.
\end{enumerate}

\section{Метрики оценки}

Для оценки качества синтезированного аудио будут использованы следующие метрики:
\begin{itemize}
    \item \textbf{PESQ} (Perceptual Evaluation of Speech Quality) — метрика качества речи.
    \item \textbf{STOI} (Short-Time Objective Intelligibility) — измеряет разборчивость речи.
    \item \textbf{MOS} (Mean Opinion Score) — субъективная оценка качества.
    \item \textbf{Визуализация спектрограмм} — для проверки соответствия и восстановления структуры спектра.
\end{itemize}

\section{Ожидаемые результаты}

Ожидается, что в первом эксперименте HiFi-GAN обеспечит лучшее качество восстановления по сравнению с Griffin-Lim. Во втором эксперименте использование собственного декодера, адаптированного под новый формат спектрограммы, может показать сопоставимые результаты с HiFi-GAN, если модель будет правильно настроена.



















\section{Результаты экспериментов}
  - картинки со спектрограммами, сравнение
  - эксперименты
   - прямой и обратный проход, 
     - разница спектрограмм, 
     - разница сигналов, 
     - человеческое восприятие
     - с реконструкцией фазы и с сохранением
     - трансформации и устойчивость к артефактам
   - самодельный автоэнкодер 
     - с сохранением фазы
	 - с реконструкцией фазы
   - text to speech сети
     - tacotron2
	 - ?

\section{Наблюдения}
  - рекомендуемые значения параметров
  - как устроена картинка с фазами, когерентный сигнал крутится вдоль времени с частотой сигнала, его след крутится весь на той же самой частоте
  - избыточность информации в шумах

\section{Практическая ценность алгоритма}
   сферы применения алгоритма, польза от него:
    - польза для нейросетей
      - скорость инференса
      - скорость обучения
      - стабильность обучения
      - простота разработки
      - без потери качества
  - сравнение метода с другими вокодерами
    - удобство
    - гибкость
    - универсальность
    - точность
    - скорость
  - примеры работы этого алгоритма и других, сравнение, подсветить недостатки других

\section{Планы развития}
 - дальнейшая работа
   - разделение на когерентный сигнал и шум (в шуме хранится больше информации чем нужно, нужно сохранять только распределение, а не реализацию)
   - реализация быстрого алгоритма на ускорителях
   - внедрение в существующие модели text-to-speech

\section{Выводы по главе}