\Introduction

В наше время активно развиваются технологии искусственного интеллекта, помогая человеку взаимодействовать с цифровым миром в понятной и удобной ему форме. 
К таким технологиям можно отнести голосовых ассистентов, переводчиков на разные языки, ИИ-агентов, помогающих человеку выполнять различные задачи.
Также искусственный интеллект открывает человеку новые возможности для творчества с помощью технологий генерации контента, такого как изображения, музыка, пение, речь.

Одной из важных составляющих упомянутых технологий является синтез речи. 
Несколько десятилетий назад возможность озвучить текст голосом, чтобы человек мог услышать и воспринять информацию, уже была прорывом. 
Причем качество сгенерированного звука заметно отличалось от звучания естесовенной речи, иногда даже резало слух, создавая эффект зловещей долины, или просто утомляя пользователя.
Также низкое качество синтезированной речи делает затруднительным ее применение в творческих сферах, таких как музыка, фильмы и игры.

В настоящее время технологии синтеза речи сильно продвинулись. Голосовые ассистенты стали умнее и их речь стала более естственной. 
Появились нейросети, способные генерировать песни по заданному тексту или мотиву (Suno \cite{SunoAI_Bark_2023}, Riffusion \cite{Riffusion}). 
Но следы того, что музыка или речь создана роботом, все еще сохраняются.
Кроме того, использование моделей ИИ для генерации звука требует использования вычислительной мощности, которая может быть недоступна некоторым мобильным устройствам.
Также существуют проблемы с вычислительной сложностью систем генерации речи в реальном времени (синхронный перевод, замена голоса). 
Они могут быть ограничены в вычислительной мощности, особенно если рассчитанны на широкую аудиторию пользователей с их устройствами.
Поэтому на данном этапе развития продолжается борьба за качество и скорость синтеза речи.

Задачи анализа и синтеза речи включают в себя кодирование/декодирование звукового сигнала в признаковое представление, более удобное для нейросети.
Этот процесс может происходить по-разному в зависимости от архитектуры нейросети и может быть источником ошибок и артефактов, либо быть вычислительно сложным.
В данной работе предлагается алгоритм кодирования звуковых данных (вокодер), который 
\begin{itemize}
    \item восстанавливает звук без потерь, 
    \item является вычислительно эффективным, по сравнению с нейросетевыми вокодерами,
    \item эффективен с точки зрения плотности полезной информации (использует Mel-шкалу частот \cite{MelScale}),
    \item устойчив к искажениям, порождаемым нейросетями, а также к некоторым полезным трансформациям
\end{itemize}
Эти свойства позволяют повысить соотношение качество/сложность в современных решениях синтеза речи.

Целью данной работы является разработка и теоретическое обоснование алгоритмического вокодера для задач синтеза речи, который будет обеспечивать высокое качество генерации звука в сочетании с нейросетями, при этом являясь вычислительно быстрым.
