\chapter*{\centering Аннотация}

% - аннотация (не более 1500 знаков)
%   - цели и задачи работы
%   - полученные результаты
%   - рекомендации, предложенные на основании данной работы

\textbf{Цель работы} - разработать алгоритм для построения спектрограммы и преобразования ее обратно в аудиосигнал, 
обеспечивающий высокое качество синтезированного звука в сочетании с популярными нейросетевыми моделями. 
Для этого алгоритм должен быть устойчив к искажениям, вносимым нейросетями. 
Также алгоритм должен быть вычислительно эффективным по сравнению с нейросетевыми декодерами (напр. HiFi-Gan)

\textbf{Задачи работы}:
\begin{itemize}
 \item Поиск подходящих методов и разработка алгоритма, эксперименты с различными методами
 \item Проверка качества звука, получаемого в сочетании с классической нейросетью для синтеза речи Tacotron2
 \item Теоретическое обоснование алгоритма
\end{itemize}

\textbf{Полученные результаты}:
\begin{itemize}
 \item При помощи метрики MOS, а также аудио- и визуального сравнения было установлено, что качество звука, декодированного разработанным алгоритмом, не хуже чем у современного нейросетевого декодера (HiFi-Gan) и заметно лучше, чем у существующего на данный момент алгоритмического декодера (GriffinLim)
 \item Разработанный алгоритм считается аналитически и не требует существенных вычислительных ресурсов по сравнению с нейросетевым декодером.
\end{itemize}

