\Conclusion % заключение к отчёту

В данной работе была рассмотрена актуальная задача повышения эффективности и доступности технологий синтеза речи. 
С развитием ИИ-систем и увеличением спроса на голосовые интерфейсы, важность качественного и быстрого синтеза речи постоянно растет. 
Особенно это касается мобильных и встроенных устройств, где ресурсы ограничены, а потребность в естественном звуке остается высокой.

На основе анализа существующих решений было выявлено, что современные нейросетевые вокодеры обеспечивают высокое качество генерации звука, 
но требуют значительных вычислительных ресурсов. 
В то же время алгоритмические методы пока уступают по качеству и устойчивости, особенно в условиях, 
когда спектрограмма порождается нейросетевой моделью и может содержать искажения.

В этой работе был предложен алгоритмический подход к декодированию спектрограмм, 
основанный на вейвлет-преобразовании с произвольной шкалой частот, включая мел-шкалу. 
Алгоритм позволяет точно восстанавливать сигнал, устойчив к искажениям, возникающим в процессе генерации, 
и может применяться в реальных TTS-системах как замена нейросетевому декодеру. 
Предложенные модификации алгоритма позволяют повысить вычислительную эффективность, 
а возможность использования алгоритма в мел-шкале делает его удобным для интеграции с существующими архитектурами.

Результаты экспериментов подтверждают, что алгоритм демонстрирует хорошее качество синтеза как в простых тестах, 
так и в сочетании с нейросетевыми моделями. 
Это открывает путь к созданию более легких, быстрых и доступных систем синтеза речи, 
без потерь в качестве звучания.

Таким образом, работа вносит вклад в развитие технологий синтеза речи, предлагая новое решение, 
которое может найти широкое практическое применение в различных областях — от голосовых ассистентов до генеративного творчества.